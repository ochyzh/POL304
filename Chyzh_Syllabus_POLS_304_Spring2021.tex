\documentclass[12pt]{article}


%%% PACKAGES
\usepackage{natbib}
\usepackage{bibentry} %to use intext full bibliography entries instead of citations.  You will need a separate BibTex database for this to work.  See http://cst.usc.edu/services/tel/grants/legrants.html for details on this package.
\usepackage{booktabs} % for much better looking tables
\usepackage{array} % for better arrays (eg matrices) in maths
\usepackage{paralist} % very flexible & customisable lists (eg. enumerate/itemize, etc.)
%\usepackage{verbatim} % adds environment for commenting out blocks of text & for better verbatim
%\usepackage{subfigure} % make it possible to include more than one captioned figure/table in a single float
\usepackage{hyperref}
\usepackage{breakurl}

%%% PAGE DIMENSIONS
\usepackage{geometry} % to change the page dimensions. Read ftp://ftp.tex.ac.uk/tex-archive/macros/latex/contrib/geometry/geometry.pdf for detailed page layout information 
\geometry{margin=1in} % for example, change the margins to 1 inches all round
%\geometry{landscape} % set up the page for landscape
% 

%%% HEADERS & FOOTERS
\usepackage{fancyhdr} % This should be set AFTER setting up the page geometry
\pagestyle{fancy} % options: empty , plain , fancy
\renewcommand{\headrulewidth}{0.4pt} % customise the layout...
%\lhead{}\chead{}\rhead{}
%\lfoot{}\cfoot{\thepage}\rfoot{}

\rfoot{\footnotesize POL 304H1-S  }
\rhead{\footnotesize Using Data to Understand Politics and Society}
\renewcommand\footrulewidth{0pt}


%%% SECTION TITLE APPEARANCE
%\usepackage{sectsty}
%\allsectionsfont{\sffamily\mdseries\upshape} % (See the fntguide.pdf for font help)
% (This matches ConTeXt defaults)


%% END Article customise

%%% BEGIN DOCUMENT


\begin{document}


\thispagestyle{plain} %alternatively specify empty to get no footer on first page.  This is part of the fancyhdr package

\bibliographystyle{plain}
\nobibliography{course_lit} %this specifies the BibTex directory that stores your desired bibliography entries.  It has to come before any \bibentry lines are invoked



%\tableofcontents

\begin{center}
\bigskip
\large{\bf{POL 304H1-S: Using Data to Understand Politics and Society}}

\textsc{Winter 2021} \\
\end{center}
\textsc{Prerequisites:} POL232H1/POL242Y1/POL322H1/equivalent

\bigskip
\noindent \textbf{Lecture:} Monday, 5--7 pm, online--synchronous \\
\noindent\textbf{Instructor: } Professor Olga Chyzh, olga.chyzh@utoronto.ca\\ 
\noindent\textbf{Office Hours:}  by appointment, \url{https://chyzh.youcanbook.me/} \\

%\noindent\textbf{Teaching Assistants: }
%\begin{itemize} 
%\item Omer Faruk Yalcin, ofyalcin@umich.edu,\\ Office Hours: \texttt{https://omerfyalcin.youcanbook.me}
%\item Nicole Schwitter, nschwitt@umich.edu,\\ Office Hours: \texttt{https://nicoleschwitter.youcanbook.me}
%\end{itemize}

\section*{Overview and Objectives}
Students will learn about role of data for understanding politics and society, and a wide range of approaches to analyzing such data. How do information and data shape politics and policies? What insights can we gain about contemporary and past societies from the data they produce? To answer these questions we will examine a series of methodological approaches to different types of political data, including text analysis, network analysis, spatial statistics, and time-series analysis. This course will draw from topics in the study of international relations, economics, political behavior, and statistics to offer a diverse set of tools for processing and analyzing different types of data. Applications will include war and conflict, terrorism, international trade, social media, elections, and representation.

\subsection*{Learning Outcomes}
This course is designed as a series of weekly modules that loosely build upon each other. Each module covers one or more state-of-the-art approaches to statistical data analysis. For each topic covered, the objectives are that students will:

\begin{itemize}
\item Learn the general mechanics of the specific method; 
\item Formulate theories and derive hypotheses testable using this method;
\item Apply the method to extract/analyze real-world political and social data.

\end{itemize}

\section*{Course Materials}

Materials for the course, including course videos, are posted on Quercus. Please allow time for processing videos.

\section*{Software}
R (latest version) \url{https://www.r-project.org/}\\
RStudio (latest version) \url{https://rstudio.com/products/rstudio/download/}

\section*{Reading}
Please complete all assigned readings prior to class.

%\section*{Homework}
%Students will apply the tools/methods covered that week to a dataset/materials made available by the instructor. Each student will submit their homework as an Rmd and an hlml file. Each homework will take 1--5 hours to complete. 

\section*{Coding Sessions}
%For each method covered we will run through applications in R during class. Students are strongly encouraged to bring their laptops to class. Students will be provided with data, but may also use their own datasets.
For each method covered we will run through applications in R during class. Students are strongly encouraged to follow along during class and review/run through these examples after class. Students will be provided with data, but may also use their own datasets.

\section*{Grading Scale}
Students will demonstrate their mastery of the course materials by completing three mini-projects (each making up 15\% of the final grade) and a final project (40\%). The remaining 15\% will come from participation. If you are unable to attend/participate in the class discussion synchronously, please contact the instructor for an alternative assignment \textbf{during the first week of class} (a failure to do so will result in a score of 0 on the participation component). No late assignments are accepted. Students who are experiencing extenuating circumstances that may prevent them from completing an assignment should contact the instructor as soon as possible.  The final grade will be calculated using the following grading scheme.\\

\begin{tabular}{ll}
A+ & $\geq 90$   \\
A & $\geq 85$  \\
A- & $\geq 80$  \\
B+ & $\geq 77$   \\
B & $\geq 73$   \\
B- & $\geq 70$   \\
C+ & $\geq 67$   \\
C & $\geq 63$   \\
C- & $\geq 60$   \\
D+ & $\geq 57$   \\
D & $\geq 53$   \\
D- & $\geq 50$   \\
F & $< 50$ \\
\end{tabular}

\subsection*{Course Policies}
\textit{Student Responsibilities in the Learning Process:} Students are expected to complete any assigned readings prior to completing that topic's assessment. Students are also expected to complete all assessments on time. This means accessing the materials with sufficient time to complete assessments prior to deadlines. In the event that a student has questions concerning the material, they should formulate specific questions to ask the professor via office hours or email with sufficient time for a response prior to assessment deadlines (i.e. emailed questions should be sent at least 24 hours prior to a deadline, excluding weekends).\\


\begin{sloppypar}
\textit{Classroom Conduct:} Students are expected to participate in class in a thoughtful and respectful manner while in the pursuit of knowledge accumulation. Generally, this means engaging with one another's ideas and treating others as you would like to be treating as well as \textit{not} treating others how you would \textit{not} like to be treated. Please see university policies on freedom of speech (\burl{https://governingcouncil.utoronto.ca/secretariat/policies/freedom-speech-statement-may-28-1992}) and discrimination and harassment (\burl{https://governingcouncil.utoronto.ca/secretariat/policies/harassment-statement-prohibited-discrimination-and-discriminatory-harassment}).
\end{sloppypar}

 \\ 

\textit{Accommodations:} Please discuss any special needs with the instructor start of the semester, for example, to request reasonable accommodations if an academic requirement conflicts with your religious practices and/or observances. Those seeking accommodations based on disabilities should complete the appropriate documentation with Student Life Programs and Services (\burl{https://studentlife.utoronto.ca/department/accessibility-services/}). \\ 


\begin{sloppypar}
\textit{Academic Misconduct:}
		All acts of dishonesty in any work constitute academic misconduct. The Student Disciplinary Regulations (\burl{https://governingcouncil.utoronto.ca/secretariat/policies/code-behaviour-academic-matters-july-1-2019}) will be followed in the event of academic misconduct.
	
\\
\end{sloppypar}

\noindent A special note on plagiarism: plagiarism is the act of representing directly or indirectly another person's work as your own. It can involve presenting someone's speech, wholly or partially, as your; quoting without acknowledging the true source of the quoted material; copying and handing in another person's work with your name on it; and similar infractions. Even indirect quotations, paraphrasing, etc., can be considered plagiarism unless sources are properly cited.\\

\item \textit{Copyright:} Course materials, including recorded lectures and slides, are the instructor's intellectual property covered by the Copyright Act, RSC 1985, c C-42. Course materials posted on Quercus or the class website may not be posted to other websites or media without the express permission of the instructor. Unauthorized reproduction, copying, or use of online recordings will constitute copyright infringement.
\newpage
\section*{Course Schedule}

\subsection*{Week 1: Introduction to R, Rstudio, and RMarkdown}

\subsection*{Week 2: Text Analysis}
	\begin{itemize}
	\item \bibentry{ArcherJockers:2016}.
	\end{itemize}
\subsection*{Week 3: Web-scraping}
\begin{itemize}
	\item Mini-Project 1 is due on Friday, January 29, at 5 pm.
		\end{itemize}

\subsection*{Week 4: Big Data}
\begin{itemize}
\item \bibentry{BoydCrawford:2011}.
\item \bibentry{Lazeretal:2014}.
\item \bibentry{Caliskanetal:2017}.
\item \bibentry{West:2010}.
	\end{itemize}
\subsection*{Week 5: Networks I: Theory}
\begin{itemize}
	\item \bibentry{Granovetter:1973}, \textbf{Trigger Warning:} This article uses racial and gendered language common at the time of its writing.
	\item \bibentry{PadgettAnsell:1993}.
	\item Apply for a Twitter developer account: \url{https://developer.twitter.com/en}. Click ``Apply'' in the top right corner and fill out the information to the best of your ability. 
	\end{itemize}

\subsection*{Week 6: Networks II: Social Media Applications}
\begin{itemize}
\item Mini-Project 2 is due on Friday, February 26, at 5 pm.
\end{itemize}

\subsection*{Week 7: Spatial Analysis I: Maps}

\subsection*{Week 8: Spatial Analysis II: Modeling Spatial Dependence}
\begin{itemize}
\item \bibentry{Becketal:2006}.
\end{itemize}

\subsection*{Week 9: Analysis of Temporal Data}
\begin{itemize}
\item \bibentry{CampbellRoss:1968}.
\item \bibentry{LewisBeck:1986}.
\item \bibentry{ChyzhUrbatsch:2021}
\item Mini-Project 3 is due on Friday, March 19, at 5 pm.
\end{itemize}

\subsection*{Week 10: Survival Analysis}
\begin{itemize}
\item \bibentry{Richetal:2010}.
\end{itemize}

\subsection*{Week 11: Experiments}
\begin{itemize}

\item \bibentry{Wilsonetal:2010}.
\item \bibentry{Aronsonetal:1999}.
\end{itemize}

\subsection*{Week 12: Instrumental Variables}
\begin{itemize}
\item \bibentry{PotoskiUrbatsch:2017}.
\item \bibentry{RitterConrad:2016}.
\item Final Project is due on Friday, April 9, at 5 pm.
\end{itemize}

\end{document} 


- Introduction. 
- Bias: Means. Simpson's paradox, Biases, Age of Legislators Exercise

- Text Analysis, Authorship of the Bible, Beatle songs
- Web-scraping
- Big Data, Google Flu example?
- Networks I: theory. Medici?  Six Degrees?
- Networks II: applications. Six Degrees? Twitter?
- Spatial Analysis I: Maps 
- Spatial Analysis II: Spatial X, spatial Y, and spatial error.
- Temporal Analysis. Discontinuity designs?  Speeding example can also go here
- Survival analysis. Kaplan Meyer.
- Experiments
- Instruments. Do Mountains Cause Civil Wars?

 
